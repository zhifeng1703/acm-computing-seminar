% Created 2017-08-28 Mon 18:17
\documentclass[11pt]{article}
\usepackage[utf8]{inputenc}
\usepackage[T1]{fontenc}
\usepackage{fixltx2e}
\usepackage{graphicx}
\usepackage{longtable}
\usepackage{float}
\usepackage{wrapfig}
\usepackage{rotating}
\usepackage[normalem]{ulem}
\usepackage{amsmath}
\usepackage{textcomp}
\usepackage{marvosym}
\usepackage{wasysym}
\usepackage{amssymb}
\usepackage{hyperref}
\tolerance=1000
\setcounter{secnumdepth}{0}
\author{Frank-PC}
\date{\today}
\title{syllabus}
\hypersetup{
  pdfkeywords={},
  pdfsubject={},
  pdfcreator={Emacs 24.5.1 (Org mode 8.2.10)}}
\begin{document}

\maketitle
cd\#+title: MAT5939-03 ACM Computing Seminar – Fall 2017

\section*{Course Information}
\label{sec-1}

\begin{center}
\begin{tabular}{ll}
\textbf{Time \& Place:} & Monday 18:45-20:00, LOV 102\\
\textbf{Instructor:} & Zhifeng Deng\\
\textbf{Office Hours:} & Tues/Thurs 12:00-13:30 or by appointment\\
\textbf{Email:} & \href{mailto:zdeng@math.fsu.edu?subject=MAT5939 ... }{zdeng@math.fsu.edu}\\
\textbf{Website:} & \href{./}{www.math.fsu.edu/\textasciitilde{}zdeng/acm-computing-seminar/}\\
\end{tabular}
\end{center}

\section*{Course Objectives}
\label{sec-2}

To learn the C++ programming langauge with a focus on  mathematical 
and scientific programming applications.

\section*{Prerequisites}
\label{sec-3}

Graduate standing in mathematics.

\section*{Course Outline}
\label{sec-4}

\begin{enumerate}
\item Getting started
\begin{itemize}
\item compilers, editors
\item writing a simple program
\end{itemize}
\item Data types
\begin{itemize}
\item common data types, casting, arrays, etc.
\end{itemize}
\item Control Structures
\begin{itemize}
\item conditionals
\item structures for iterating
\end{itemize}
\item Input and output
\begin{itemize}
\item arguments from the command line
\item reading and writing to files
\end{itemize}
\item Functions
\begin{itemize}
\item writing functions
\item modularizing code with header and implementation files
\item function pointers
\end{itemize}
\item Object-oriented programming
\begin{itemize}
\item basics of writing classes: constructors, destructors, operator overloading, etc.
\end{itemize}
\end{enumerate}

\section*{Grade Policy:}
\label{sec-5}

Homework is assigned weekly and counts for 50\% of your final grade.

4 comprehensive programs will be released in the second week and they count for
the other 50\% of your grade. You can submit your programs anytime before the seminar
ends.

Your final grade will be $(Average(HW)+Average(Programs)/2)$

Homework and programs are graded on a 0 – 3 scale:

\begin{itemize}
\item 0 - no result submitted or code copied from another source
\item 1 - result submitted, but did not demonstrate code functions properly
\item 2 - showed code functions properly, but results not fully explained
\item 3 - showed code functions properly, presented results clearly and fully
\end{itemize}

To earn a grade of S, you must have final grade greater or equal to 2. 

For each homework or program assignment you will hand in a hard-copy report and e-mail me 
your source code. 

For each program assignment, you are expected to come to my office and run your code with different
inputs.

When e-mailing, please include [MAT5939] in the subject line 
of the email.

\section*{Homework Policy}
\label{sec-6}

You must convince me your code functions as required. It is up to you to 
consider sufficient test cases, although I will provide guidance and more 
examples of what this means in the first couple of assignments. The reports 
should be considerably less detailed than those submitted for FCM. For example, 
there is no need for a "Description of Mathematics" or 
"Decription of Algorithm"; rather, an explanation of why you chose the 
test cases and any special programming methods/data structures you decided 
to use to solve the problem will be sufficient.

\section*{Attendance Policy}
\label{sec-7}

Excused absences include documented illness, deaths in the family and other 
documented crises, call to active military duty or jury duty, religious holy 
days, and official University activities. These absences will be accommodated 
in a way that does not arbitrarily penalize students who have a valid excuse. 
Consideration will also be given to students whose dependent children 
experience serious illness.

\section*{Academic Honor Policy}
\label{sec-8}

The Florida State University Academic Honor Policy outlines the University’s 
expectations for the integrity of students’ academic work, the procedures for 
resolving alleged violations of those expectations, and the rights and 
responsibilities of students and faculty members throughout the process. 
Students are responsible for reading the Academic Honor Policy and for living 
up to their pledge to “\ldots{} be honest and truthful and \ldots{} [to] strive for 
personal and institutional integrity at Florida State University.” 

(Florida State University Academic Honor Policy, found at
\url{http://fda.fsu.edu/Academics/Academic-Honor-Policy}.)

\section*{Americans with Disabilities}
\label{sec-9}

Students with disabilities needing academic accommodation should:

\begin{enumerate}
\item Register with and provide documentation to the Student Disability Resource Center
\item Bring a letter to the instructor indicating the need for accommodation and what type. This should be done
\end{enumerate}

during the first week of class.

This syllabus and other class materials are available in alternative format 
upon request. For more information about services available to FSU students 
with disabilities, contact the:

Student Disability Resource Center
874 Traditions Way
108 Student Services Building
Florida State University
Tallahassee, FL 32306-4167
(850) 644-9566 (voice)
(850) 644-8504 (TDD)

sdrc@admin.fsu.edu
\url{http://www.disabilitycenter.fsu.edu/}

\section*{Syllabus Change Policy}
\label{sec-10}

Except for changes that substantially affect implementation of the evaluation (grading)
statement, this syllabus is a guide for the course and is subject to change with advance notice.
% Emacs 24.5.1 (Org mode 8.2.10)
\end{document}
